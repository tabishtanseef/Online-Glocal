\documentclass[a4paper,8pt]{article}
\usepackage{amssymb,amsthm}
\usepackage{amsmath}
\usepackage{setspace}
\usepackage{graphicx}
\usepackage{qtree}
\usepackage{xcolor}
\usepackage{tikz}
\usetikzlibrary{automata,positioning,decorations.markings,arrows}
\begin{document}
 Q8. Consider a network connected two systems located 9000 kilometers apart. and transmission-
  time of 120ms . 7 bits are used for the sequence no field .\\\\
(a) Assume that processing delays at nodes are negligible. And network Utilization is
70 \% . Then Calculate average speed of the signal on the network if following flow
control technique is used.\\\\
i. Go-Back-N\\
ii. Selective Repeat\\
iii. Stop and Wait\\\\
(b) Considering the average speed of the signal on the network , we have calculated in
part-a , calculate the throughput of the system in case of Go-Back-N and Selective
Repeat .\\\\
(c) Assume that processing delays at receiver node is 100ms. And network Utilization
is 50 \% . Then Calculate average speed of the signal on the network if following flow
control technique is used\\\\
i. Go-Back-N\\
ii.Selective Repeat\\
iii.Stop and Wait\\\\
(d) Considering the frame that we have calculated in part-c , calculate the throughput
of the system in case of Go-Back-N and Selective Repeat .\\\\
(e) Assume that processing delays at nodes are negligible. And network Utilization is
100 \% . Then Calculate average speed of the signal on the network if following\\\\
flow control technique is used\\\\
i. Go-Back-N\\
ii. Selective Repeat\\
iii. Stop and Wait\\\\

(a) Assume that processing delays at nodes are negligible. And network Utilization is
80 \% . Then Calculate the minimum size in bits of the sequence number field if following 
flow control technique is used\\\\
i. Go-Back-N\\
ii.Selective Repeat\\\\

distance=9000km\\\\
$t_{t}=120ms$\\\\
n=7 bits\\\\
u=70\%  =$\frac{70}{100}$\\\\
go-back N\\\\
U=$\frac{W*t_{t}}{t_{t}+2t_{p}}$\\\\
w=$2^{n}-1$\\\\
=$2^{7}-1  =128-1=127$\\\\
$\frac{7}{10}$=$\frac{127*120}{120+2t_{p}}$\\\\
$7(120+2t_{p})=10(127*120)$\\\\
$840+14t_{p}=152400$\\\\
$14t_{p}=152400-840$\\\\
$t_{p}=\frac{151560}{14}=10825.7 ms$\\\\
$t_{p}=\frac{10825.7}{10^{-3}}$\\\\
$t_{p}=10.8 sec$\\\\
$\frac{distance}{speed}$=10.8\\\\
$speed=\frac{9000}{10.8}$\\\\
$speed=833.3 km/sec$\\\\
selective repeat-\\\\
$w=2^{n-1}=2^{7-1}=2^{6}=64$\\\\
$u=t_{t}*\frac{w}{t}_{t}+2t_{p}$\\\\
$\frac{7}{10}=\frac{64*120}{120+2t_{p}}$\\\\
$7(120+2t_{p})=10(64*120)$\\\\
$840+14t_{p}=76800$\\\\
$14t_{p}=76800-840$\\\\
$t_{p}=\frac{75960}{14} =5425.7 ms$\\\\
$t_{p}=5425.7*10^{-3}=5.4 sec$\\\\
$\frac{distance}{speed}=5.4 sec$\\\\
$speed=\frac{9000}{5.4}$\\\\
$speed=1666.6 km/sec$\\\\

(b) Considering the average speed of the signal on the network , we have calculated in
part-a , calculate the throughput of the system in case of Go-Back-N and Selective
Repeat .\\\\

GO-BACK-N\\\\
$T=\frac{W*D}{t_{t}+2t_{p}}$\\\\
$t=\frac{127*10^{7}}{120+2*10825.7}$\\\\
$t=\frac{127}{21771.4}$\\\\
$t=0.00583341*10^{7} bits$\\\\
$t=58334.1 bits$\\\\
SELECTIVE REPEAT-\\\\
$t=\frac{64*d}{t_{t}+2t_{p}}$\\\\
$t=\frac{64*10^{7}}{21771.4}=0.0029396364*10^{7}=29396.36 bits$\\\\

(c) Assume that processing delays at receiver node is 100ms. And network Utilization
is 50 \% . Then Calculate average speed of the signal on the network if following flow
control technique is used\\\\
i. Go-Back-N\\
ii. Selective Repeat\\
iii. Stop and Wait\\\\

GO-BACK-N-
$U=\frac{w*t_{t}}{t_{t}+2t_{p}}$\\\\
$\frac{50}{100}=\frac{127*120}{220+2t_{p}}$\\\\
$\frac{1}{2}=\frac{127*120}{220+2t_{p}}$\\\\
$220+2t_{p}=254+240$\\\\
$220+2t_{p}=274$\\\\
$t_{p}=\frac{274}{2}=137 ms =137*10^{-3}$\\\\
$t_{p}=.14 sec$\\\\
$\frac{distance}{speed}=.14$\\\\
$\frac{9000}{.14}=750 km/sec$\\\\

SELECTIVE REPEAT-
$W=64$\\\\
$U=\frac{W*t_{t}}{t_{t}+2t_{p}}$\\\\
$\frac{1}{2}=\frac{120*64}{220=2t_{p}}$\\\\
$220+2t_{p}=240*128$\\\\
$220+2t_{p}=30720$\\\\
$2t_{p}=30500$\\\\
$t_{p}=306*10^{-3}$\\\\
$t_{p}=.30$\\\\
$\frac{distance}{speed}=.30$\\\\
$speed=\frac{9000}{.30}$\\\\
$speed=300 km/sec$\\\\

STOP AND WAIT-
$U=\frac{t_{t}}{t_{t}+2t_{p}}$\\\\
$\frac{50}{100}=\frac{120}{220+2t_{p}}$\\\\
$\frac{1}{2}=\frac{120}{220+2t_{p}}$\\\\
$220+2t_{p}=240$\\\\
$2t_{p}=240-220$\\\\
$t_{p}=\frac{20}{2}=10$\\\\
$t_{p}=10*10^{-3}$\\\\
$t_{p}=\frac{distance}{speed}$\\\\
$speed=\frac{9000}{.010}=9000 km/sec$\\\\
(d) Considering the frame that we have calculated in part-c , calculate the throughput
of the system in case of Go-Back-N and Selective Repeat .\\\\

GO-BACK-N-
$t=\frac{w*d}{t_{t}+2t_{p}}$\\\\
$t=\frac{127*10^{6}*10}{220+2*.12}$\\\\
$t=\frac{1270}{220+.24}$\\\\
$t=\frac{1270*10^{6}}{220.24}$\\\
$t=5.76*10^{6}$\\\


SELECTIVE REPEAT-
$T=\frac{w*d}{t_{t}+2t_{p}}$\\\
$t=\frac{64*10*10^{6}}{220+2*.30}$\\\
$t=\frac{640*10^{6}}{220+.60}$\\\
$t=\frac{640*10^{6}}{220.60}$\\\\
$t=2.9*10^{6}$\\\

(e) Assume that processing delays at nodes are negligible. And network Utilization is
100 \% . Then Calculate average speed of the signal on the network if following flow
control technique is used\\\\
i. Go-Back-N\\
ii. Selective Repeat\\
iii. Stop and Wait\\\\
GO-BACK-N\\
$U=\frac{w*t_{t}}{t_{t}+2t_{p}}$\\\\
$w=2^{n}-1$\\\\
$w=2^{7}-1=127$\\\\
$\frac{100}{100}=\frac{127*120}{120+2t_{p}}$\\\\
$1=\frac{127*120}{120+2t_{p}}$\\\\
$120+2t_{p}=127*120$\\\\
$120+2t_{p}=15240$\\\\
$2t_{p}=15120$\\\\
$t_{p}=7560=7560*10^{-3}$\\\\
$t_{p}=.756$\\\\
$\frac{distance}{speed}=t_{p}$\\\\
$\frac{distance}{speed}=.756$\\\\
$speed=\frac{9000}{.756}=11904.8 km/sec$\\\\

SELECTIVE REPEAT-
$U=\frac{W*t_{t}}{t_{t}+2t_{p}}$\\\\
$w=2^{n-1}=2^{7-1}$\\\\
$w=64$\\\\
$\frac{100}{100}=\frac{64*120}{120+2t_{p}}$\\\\
$1=\frac{64*120}{120+2t_{p}}$\\\\
$120+2t_{p}=64*120$\\\\
$2t_{p}=768-120$\\\\
$t_{p}=\frac{648}{2}=324$\\\\
$t_{p}=324*10^{-3}=.324 km/sec$\\\\
$\frac{distance}{speed}=t_{p}$\\\\
$\frac{distance}{speed}=.324$\\\\
$speed=\frac{9000}{.324}$\\\\
$speed=27777.8 km/sec$\\\\

STOP AND WAIT-
$U=\frac{t_{t}}{t_{t}+2t_{p}}$\\\\
$\frac{100}{100}=\frac{120}{120+2t_{p}}$\\\\
$120+2t_{p}=120$\\\\
$2t_{p}=120-120$\\\\
$t_{p}=0$\\\\
$t_{p}=\frac{distance}{speed}$\\\\
$speed=\frac{9000}{0}=0 km/sec$\\\\




















\end{document}